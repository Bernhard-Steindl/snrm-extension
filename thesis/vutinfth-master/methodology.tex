\chapter{Methodology}

% Intro:
In this chapther the approach of realizing this bachelor thesis is illustrated.
It answers what the theoretical and practical values of the work are.
Also, an original plan of this thesis' goals and tasks is shown and compared with the
    actual, reached objectives.

% Overview practical values
The practical values of the bachelor thesis is the reproducibility of
    the SNRM implementation and the migration of the original SNRM source 
    code, that was written with the TensorFlow
    \footnote{TensorFlow website \url{https://www.tensorflow.org}}
    library, 
    to the currently latest version of the machine learning
    library PyTorch \footnote{PyTorch website \url{https://pytorch.org}}.
Instead of using the same datasets as SNRM's authors,
    Microsoft's MAchine Reading COmprehension Dataset (MS MARCO)
    \footnote{MS MARCO website \url{http://www.msmarco.org}}
    for passage ranking is used for this thesis,
    due to its free availability for non-commercial research purposes.

% Analysis machine-learning popularity
An analysis from Horace He \cite{he:2019:state-of-ml-frameworks} shows that 
    in 2019 the machine learning frameworks PyTorch and TensorFlow 
    are the most popular used ones.
According to Horace He, TensorFlow is still the dominant framework
    in the industry by comparing job listings, for instance.
However, he also studied research papers that either use PyTorch or
    TensorFlow and found PyTorch has gained traction since 2018 and
    actually holds the majority in the research community at the moment.
    \cite{he:2019:state-of-ml-frameworks}\\
The growing popularity of using PyTorch motivates the usage of the 
    framework for this bachelor thesis.

% Overview theoretical values
The theoretical values of the bachelor thesis is a comparison of the 
    implementation design of the original SNRM and the migrated SNRM, 
    as well as an analysis of state of the art neural information retrieval models,
    like the Standalone Neural Ranking Model (SNRM) 
    from Zamani et al. \cite{zamani:2018:from-neural-reranking-to-neural-ranking} 
    and the Neural Vector Space Model (NVSM) 
    from Van Gysel et al. \cite{van-gysel:2017:neural-vector-spaces}.

% Overview-Detail practical values
Concerning the practical part of the thesis, the first goal was to be able 
    to run the original SNRM.
For training and evaluating the IR model the MS MARCO dataset
    for passage ranking (MSMARCO-Passage-Ranking) is used.
One of the first steps was to find compatible
    versions for Python 
    \footnote{Python website \url{https://www.python.org}}, 
    TensorFlow and NumPy
    \footnote{NumPy website \url{https://www.numpy.org}},
    because the originally used legacy package versions of the authors are not
    listed on the GitHub page of SNRM 
    \footnote{Hamed Zamani's SNRM github repository \url{https://github.com/hamed-zamani/snrm}}.
Next was the implementation of missing Python functions in the original SNRM code, 
    comprising among others the creation of an in-memory term dictionary, 
    and an extension for provisioning of training and evaluation batch data from the 
    MSMARCO dataset to the model.
Model trainings and evaluations were carried out in the course of experiments 
    in order to try to reproduce the proposed retrieval results by the authors of SNRM.\\
Afterwards, the original SNRM was migrated to a PyTorch implementation,
    which can be imagined as a SNRM with currently up-to-date versions of 
    Python, PyTorch and NumPy.
The PyTorch implementation was also accompanied by model evaluations and experiments.
For version control of the source code a private GitHub repository, forked from 
    the original SNRM, was used.\\
The design of the model implementations, as well as 
    challenges that have arisen in accompanying experiments and associated results 
    are available in the 
    thesis chapter \nameref{chapter:design-and-implementation}.

% Overview-Detail practical values
For the further theoretical part, a comparison, as well as an evaluation of the model's
    training performance and its ranking results of the original SNRM and 
    the PyTorch implementation with the MS MARCO dataset 
    were planned. 
However, these planned evaluations were revoked,
    due to the encountered challenges during the implementation and the experiments.
Instead, the challenges that arose during experiments 
    and corresponding approaches to solving them are depicted in the thesis.\\
Additionally, in this work's chapter
    \nameref{chapter:state-of-the-art} 
    is a comparison of the Standalone Neural Ranking Model to other 
    recent proposals of neural information retrieval models,
    like the Neural Vector Space Model (NVSM).

% Original Methodology / Plan according to thesis proposal
Before working on the thesis, a thesis proposal was written for
    defining the scope and the objectives of the work.
The following list shows the original planned tasks of the thesis, 
    which were approved by the advisor assistance.
\begin{enumerate}
    \item research and study literature
    \item write and submit bachelor thesis proposal
    \item extend original SNRM to run it with MS MARCO passage ranking dataset
    \item study of PyTorch and TensorFlow
    \item technical design of SNRM-PyTorch
    \item implementation of SNRM-PyTorch
    \item comparison and evaluation of SNRM implementations
    \item write bachelor thesis
    \item final design and code review of the implementation
    \item final review of the evaluation
    \item incorporation of final feedback of the thesis from the assistant
    \item proof-read the thesis
\end{enumerate}
The current progress during the course of the work was continuously
    shared and discussed with the advisor assistance.
Helpful hints and feedback from the advisor assistance were gratefully 
    incorporated in the work of the thesis.
Through frequent information on the status of the implementation 
    and the experiments, the initial objectives of the thesis were adapted, 
    with regard to the comparison and evaluation of the SNRM implementations.

Basic theoretical knowledge to work on this thesis was acquired among others 
    through a university lecture course on Introduction to Information Retrieval,
    and by reading papers
    from Zamani et al. \cite{zamani:2018:from-neural-reranking-to-neural-ranking},
    and Van Gysel et al. \cite{van-gysel:2017:neural-vector-spaces}, for instance.
In addition, the paper from Mitra et al. on 
    an Introduction to Neural Information Retrieval \cite{mitra:2018:introduction-neural-ir},
    Coursera's Stanford Machine Learning Course
    \footnote{Stanford's Machine Learning Course on Coursera's website \url{https://www.coursera.org/learn/machine-learning}},
    and Chollet's book on Deep Learning with Python \cite{chollet:2017:deep-learning-with-python}
    were helpful for getting started with the thesis.
