\chapter{Methodology}

The practical values of the bachelor thesis will be the reproducibility of
    the SNRM implementation and the migration of the original SNRM source 
    code to be executable with the latest versions of the machine learning 
    library PyTorch \footnote{PyTorch website \url{https://pytorch.org}}  

An analysis \cite{he:2019:state-of-ml-frameworks} from Horace He shows that 
    in 2019 the machine learning frameworks PyTorch and TensorFlow 
    \footnote{TensorFlow website \url{https://www.tensorflow.org}}
    are the most popular used ones.
According to Horace He, TensorFlow is still the dominant framework
    in the industry by comparing job listings, for instance.
However, he also studied research papers that either use PyTorch or
    TensorFlow and found PyTorch has gained traction since 2018 and
    actually holds the majority in the research community at the moment.
    \cite{he:2019:state-of-ml-frameworks}
The growing popularity of using PyTorch motivates the usage of the 
    framework for the bachelor thesis.

The theoretical values of the bachelor thesis will be a comparison of 
    the original SNRM and the ported SNRM, as well as an analysis 
    of state of the art neural information retrieval models, 
    like SNRM and NVSM.\\
Concerning the practical part of the thesis, the first goal is to be able 
    to run the original SNRM.
For training and evaluation of the IR model the 
    Microsoft MAchine Reading COmprehension Dataset (MS MARCO 
    \footnote{MS MARCO website \url{http://www.msmarco.org}}) 
    for passage ranking (MSMARCO-Passage-Ranking) will be used.
One of the first steps to approach this goal is to find compatible
    versions for Python \footnote{Python website \url{https://www.python.org}}, 
    TensorFlow and NumPy \footnote{Numpy website \url{https://www.numpy.org}},
    because the originally used legacy package versions of the authors are not
    listed on the GitHub page of SNRM 
    \footnote{Hamed Zamani's SNRM github repository \url{https://github.com/hamed-zamani/snrm}}.
Next up, will be the implementation of missing Python functions in the original SNRM code, 
    comprising among others the creation of an in-memory term dictionary, 
    and the provision of training and evaluation batch data to the model.
Afterwards, the original SNRM will be ported resp. migrated to SNRM-PyTorch, 
    which will be a SNRM with up-to-date versions of Python, PyTorch and NumPy.
For version control of the source code a private GitHub repository, forked from 
    the original SNRM, will be used.
The design of the implementation and the source code will be available in the thesis
    chapter Design and Implementation.

For the theoretical part, a comparison as well as an evaluation of the model 
    training performance and ranking results of the original SNRM and SNRM-PyTorch 
    with the MS MARCO dataset will be presented in the chapter Evaluation.

The bachelor thesis chapter State of the Art will be a comparison of SNRM to other 
    recent proposals of Neural Information Retrieval models, 
    like the Neural Vector Space Model (NVSM).