\chapter{Summary}

% Conclusion

In this bachelor thesis the need of neural information retrieval models was 
    depicted.
This work highlights theoretic aspects and objectives of
    two IR models that use different neural approaches.
The thesis' first theoretic part was dedicated to 
    the Neural Vector Space Model (NVSM) from Van Gysel et al.,
    that uses unsupervised learning to learn low-dimensional representations 
    of words and documents, and to learn a transformation function between
    a word and document feature space.
Besides that, the Standalone Neural Ranking Model (SNRM) from Zamani et al.
    was examined,
    that utilizes supervised learning to generate high-dimensional sparse
    represenations out of query and document text with the intention 
    to construct an inverted index for ranking.

Further, the practical focus was first to implement missing but necessary
    functionality in the original SNRM source code and to  
    extend the SNRM to work with
    Microsoft's MS MARCO passage ranking dataset.
In the next step, the original and extended version of the SNRM was
    migrated from TensorFlow version 1 to a
    version of PyTorch that was the most recent at the time of 
    implementation.
In the bachelor thesis the challenges and problems in experiments and 
    in reproducting the results of Zamani et al. were described in detail.
Unfortunately, the original SNRM source code and the paper lack 
    information concerning reproduction of the results.
Despite, making assumptions of originally used sofware versions,
    implementation details,
    chosen hyper parameters and configurations, it was neither possible
    with the extended TensorFlow nor with the PyTorch implementation of the SNRM 
    to reproduce the same or similar results of the authors 
    with the MS MARCO dataset.
Primarily, it was difficult to control the sparsity of
    learned representations, so that they are just sparse enough for
    efficient inverted index construction.
By adjusting the neural network architecture, 
    trying different configurations and adapting the loss computation
    among others,
    it was possible to generate a model with roughly sufficient sparsity,
    however the trained model was never able to learn semantic relations 
    between words in retrieval experiments, 
    and therefore an evaluation of the model's retrieval performance could not 
    be presented.
Due to the lack of a personal NVIDIA GPU for GPU-accelerated model training 
    and the limitations of Google Colaboratory, experiments were further 
    constrainted.

The migrated PyTorch implementation of the SNRM serves as a starting point
    to make it easier to keep up to date with future versions of PyTorch
    and to support those who are familiar with PyTorch but not with TensorFlow
    to reproduce the SNRM's results.
Further experiments with a different training dataset are imagineable.
Possibly, the SNRM was not able to learn semantic meaning and word relations,
    since passage text's are naturally much shorter than document texts.
It may therefore be worth considering the use of a different training dataset 
    containing longer document texts.